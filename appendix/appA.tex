% -*-coding: utf-8 -*-

\BiAppChapter{公式定理证明}{Proofs of Equations and Theorems}

附录编号依次编为附录~A,附录~B。附录中的图、表、公式另行编排序号,编号前加“附录A-”字样。

排版数学定理等环境时最好给环境添加结束符,以明确定理等内容的起止标志,方便阅读。例如定义的结束符采用~$\Diamond$,例子的结束符采用~$\blacklozenge$,定理的结束符采用~$\square$,证明的结束符采用~$\blacksquare$。

\begin{definition}[向量空间]
	设~$X$~是一个非空集合,$\mathbb{F}$~是一个数域~(实数域~$\mathbb{R}$~或者复数域~$\mathbb{C}$)。如果在~$X$~上定义了加法和数乘两种运算,并且满足以下~8~条性质:
	%
	\begin{enumerate}
		\item 加法交换律,$\forall~x,y \in X$,$x+y = y+x \in X$;
		\item 加法结合律,$\forall~x,y,z \in X$,$(x+y)+z = x+(y+z)$;
		\item 加法的零元,$\exists~0 \in X$,使得~$\forall~x \in X$,$0+x = x$;
		\item 加法的负元,$\forall~x \in X$,~$\exists~-x \in X$,使得~$x+(-x) = x-x = 0$。
		\item 数乘结合律,$\forall~\alpha,\beta \in \mathbb{F}$,$\forall~x \in X$,$(\alpha\beta)x = \alpha(\beta x) \in X$;
		\item 数乘分配律,$\forall~\alpha \in \mathbb{F}$,$\forall~x,y \in X$,$\alpha(x+y) = \alpha x + \alpha y$;
		\item 数乘分配律,$\forall~\alpha,\beta \in \mathbb{F}$,$\forall~x \in X$,$(\alpha+\beta)x = \alpha x + \beta x$;
		\item 数乘的幺元,$\exists~1 \in \mathbb{F}$,使得~$\forall~x \in X$,$1 x = x$,
	\end{enumerate}
	%
	那么称~$X$~是数域~$\mathbb{F}$~上的一个\textbf{向量空间}~(linear space)。
\end{definition}

\begin{example}[矩阵空间]
	所有~$m\times n$~的矩阵在普通矩阵加法和矩阵数乘运算下构成一个向量空间~$\mathbb{C}^{m\times n}$。如果定义内积如下:
	%
	\begin{equation}
	\iprod{A}{B} = \mathrm{tr}(B^H Q A) = \sum_{i=1}^{n} b_i^H Q a_i
	\end{equation}
	%
	其中~$a_i$~和~$b_i$~分别是~$A$~和~$B$~的第~$i$~列,而~$Q$~是~HPD~矩阵,那么~$\mathbb{C}^{m\times n}$~构成一个~Hilbert~空间。当~$Q=I$~时
	%
	\begin{equation}
	\iprod{A}{B} = \mathrm{tr}(B^H A)
	\end{equation}
	%
	称为~Frobenius~内积,对应的范数称为~Frobenius~范数,即矩阵所有元素模平方之和再开方:
	%
	\begin{equation} \label{equ_chap1_frob_norm}
	\|A\|_F = \sqrt{\mathrm{tr}(A^H A)} = \sqrt{\sum_{j=1}^{n}\sum_{i=1}^{m} |a_{ij}|^2}
	\end{equation}
	%
	如果~$m=n$,那么所有~$m\times m$~的~Hermite~矩阵构成~$\mathbb{C}^{m\times m}$~的子空间。但是所有~$m\times m$~的~HPD~矩阵并不构成子空间,因为~HPD~矩阵对线性运算不封闭。
\end{example}

\begin{theorem}[Riesz 表示定理]
	设~$H$~是~Hilbert~空间,$H^{\ast}$ 是 $H$ 的对偶空间,那么对~$\forall~f\in H^{\ast}$,存在唯一的~$x_f\in H$,使得
	%
	\begin{equation}
	f(x) = \iprod{x}{x_f}, \qquad \forall~x \in H
	\end{equation}
	%
	并且满足 $\|f\|=\|x_f\|$。
\end{theorem}

\begin{proof}
	先证存在性,再证唯一性,最后正~$\|f\|=\|x_f\|$。
\end{proof}
